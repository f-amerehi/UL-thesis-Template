\chapter{Introduction}

The introduction serves to set the stage for your research and helps the reader understand the context and significance of your work. The following sections should be included in the introduction:


\begin{itemize}
    \item \textbf{Background and Context}: Provide an overview of the broad area your research is situated in. This is where you set the context for your work and explain why this area of study is important.
    
    \item \textbf{Research Problem}: Clearly define the specific research problem your thesis addresses. Describe the gap in existing knowledge that you aim to fill with your research.
    
    \item \textbf{Objectives and Research Questions}: State the specific objectives of your research. These are often framed as research questions or hypotheses you intend to explore throughout the thesis.
    
    \item \textbf{Methodology Overview}: Provide a brief summary of your research methods. This section should give an overview of the approach you used to gather data or conduct experiments, and how these methods align with the research objectives.
    
    \item \textbf{Significance of the Study}: Explain why your research matters. Discuss the potential contributions to the field, its relevance, and any practical applications of your findings.
    
    \item \textbf{Thesis Structure}: Outline the structure of your thesis. Briefly summarize the content of each chapter so the reader knows what to expect.
\end{itemize}



\lipsum[2-7]